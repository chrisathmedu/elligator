% In this file you should put the actual content of the blueprint.
% It will be used both by the web and the print version.
% It should *not* include the \begin{document}
%
% If you want to split the blueprint content into several files then
% the current file can be a simple sequence of \input. Otherwise It
% can start with a \section or \chapter for instance.
\begin{definition}[Elligator 1 c]
  \label{def:Elligator1.c}
  \lean{Elligator1.c}
  \leanok
  TODO
\end{definition}

\begin{theorem}[c Property]
  \label{thm:Elligator1.c_h}
  \lean{Elligator1.c_h}
  \leanok
  \uses{def:Elligator1.c}
  c_of_s * (c_of_s - 1) * (c_of_s + 1) ≠ 0
\end{theorem}

\begin{definition}[Elligator 1 r]
  \label{def:Elligator1.r}
  \lean{Elligator1.r}
  \leanok
  \uses{def:Elligator1.c}
  TODO
\end{definition}

\begin{definition}[Elligator 1 d]
  \label{def:Elligator1.d}
  \lean{Elligator1.d}
  \leanok
  \uses{def:Elligator1.c}
  TODO
\end{definition}

\begin{definition}[Elligator 1 u]
  \label{def:Elligator1.u}
  \lean{Elligator1.u}
  \leanok
  TODO
\end{definition}

\begin{definition}[Elligator 1 v]
  \label{def:Elligator1.v}
  \lean{Elligator1.v}
  \leanok
  \uses{def:Elligator1.u, def:Elligator1.r}
  TODO
\end{definition}

\begin{definition}[Legendre Symbol χ]
  \label{def:LegendreSymbol.χ}
  \lean{LegendreSymbol.χ}
  \leanok
  TODO
\end{definition}

\begin{definition}[Elligator 1 X]
  \label{def:Elligator1.X}
  \lean{Elligator1.X}
  \leanok
  \uses{def:Elligator1.u, def:Elligator1.v, def:LegendreSymbol.χ}
  TODO
\end{definition}

\begin{definition}[Elligator 1 Y]
  \label{def:Elligator1.Y}
  \lean{Elligator1.Y}
  \leanok
  \uses{def:Elligator1.u, def:Elligator1.c, def:Elligator1.v, def:LegendreSymbol.χ}
  TODO
\end{definition}

\begin{definition}[Elligator 1 x]
  \label{def:Elligator1.x}
  \lean{Elligator1.x}
  \leanok
  \uses{def:Elligator1.X, def:Elligator1.c, def:Elligator1.Y}
  TODO
\end{definition}

\begin{definition}[Elligator 1 y]
  \label{def:Elligator1.y}
  \lean{Elligator1.y}
  \leanok
  \uses{def:Elligator1.r, def:Elligator1.X, def:Elligator1.Y}
  TODO
\end{definition}

\begin{definition}[Decoding Function ϕ]
  \label{def:Elligator1.ϕ}
  \lean{Elligator1.ϕ}
  \uses{def:Elligator1.x, def:Elligator1.y}
  \leanok
  TODO
\end{definition}

\begin{theorem}[Variables fulfill Specific Equation]
  \label{thm:map_fulfills_specific_equation}
  \lean{Elligator1.map_fulfills_specific_equation}
  \leanok
  \uses{def:Elligator1.c, def:Elligator1.X, def:Elligator1.Y}
  There is a homotopy of immersions of $𝕊^2$ into $ℝ^3$ from the inclusion map to
  the antipodal map $a : q ↦ -q$.
\end{theorem}

\begin{theorem}[Variables fulfill Curve Equation]
  \label{thm:Elligator1.map_fulfills_curve_equation}
  \lean{Elligator1.map_fulfills_curve_equation}
  \leanok
  \uses{def:Elligator1.x, def:Elligator1.y, def:Elligator1.d, thm:Elligator1.map_fulfills_specific_equation}
  There is a homotopy of immersions of $𝕊^2$ into $ℝ^3$ from the inclusion map to
  the antipodal map $a : q ↦ -q$.
\end{theorem}
  
\begin{proof}
  \leanok
  This obviously follows from what we did so far.
\end{proof}
