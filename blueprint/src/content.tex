% In this file you should put the actual content of the blueprint.
% It will be used both by the web and the print version.
% It should *not* include the \begin{document}
%
% If you want to split the blueprint content into several files then
% the current file can be a simple sequence of \input. Otherwise It
% can start with a \section or \chapter for instance.
\begin{definition}[Decoding Function Variable c]
  \label{def:Elligator1.c}
  \lean{Elligator1.c}
  \leanok
  TODO
\end{definition}

\begin{theorem}[c Property]
  \label{thm:Elligator1.c_h}
  \lean{Elligator1.c_h}
  \leanok
  \uses{def:Elligator1.c}
  TODO
\end{theorem}

\begin{proof}
  \leanok
  This obviously follows from what we did so far.
\end{proof}

\begin{definition}[Decoding Function Variable r]
  \label{def:Elligator1.r}
  \lean{Elligator1.r}
  \leanok
  \uses{def:Elligator1.c}
  TODO
\end{definition}

\begin{definition}[Decoding Function Variable d]
  \label{def:Elligator1.d}
  \lean{Elligator1.d}
  \leanok
  \uses{def:Elligator1.c}
  TODO
\end{definition}

\begin{definition}[Decoding Function Variable u]
  \label{def:Elligator1.u}
  \lean{Elligator1.u}
  \leanok
  TODO
\end{definition}

\begin{definition}[Decoding Function Variable v]
  \label{def:Elligator1.v}
  \lean{Elligator1.v}
  \leanok
  \uses{def:Elligator1.u, def:Elligator1.r}
  TODO
\end{definition}

\begin{definition}[Legendre Symbol χ]
  \label{def:LegendreSymbol.χ}
  \lean{LegendreSymbol.χ}
  \leanok
  TODO
\end{definition}

\begin{definition}[Decoding Function Variable X]
  \label{def:Elligator1.X}
  \lean{Elligator1.X}
  \leanok
  \uses{def:Elligator1.u, def:Elligator1.v, def:LegendreSymbol.χ}
  TODO
\end{definition}

\begin{definition}[Decoding Function Variable Y]
  \label{def:Elligator1.Y}
  \lean{Elligator1.Y}
  \leanok
  \uses{def:Elligator1.u, def:Elligator1.c, def:Elligator1.v, def:LegendreSymbol.χ}
  TODO
\end{definition}

\begin{definition}[Decoding Function Variable x]
  \label{def:Elligator1.x}
  \lean{Elligator1.x}
  \leanok
  \uses{def:Elligator1.X, def:Elligator1.c, def:Elligator1.Y}
  TODO
\end{definition}

\begin{definition}[Decoding Function Variable y]
  \label{def:Elligator1.y}
  \lean{Elligator1.y}
  \leanok
  \uses{def:Elligator1.r, def:Elligator1.X, def:Elligator1.Y}
  TODO
\end{definition}

\begin{theorem}[Decoding Function Variables fulfill Specific Equation]
  \label{thm:Elligator1.map_fulfills_specific_equation}
  \lean{Elligator1.map_fulfills_specific_equation}
  \uses{def:Elligator1.c, def:Elligator1.X, def:Elligator1.Y}
  TODO
\end{theorem}

\begin{theorem}[Decoding Function Variables fulfill Curve Equation]
  \label{thm:Elligator1.map_fulfills_curve_equation}
  \lean{Elligator1.map_fulfills_curve_equation}
  \leanok
  \uses{def:Elligator1.x, def:Elligator1.y, def:Elligator1.d, thm:Elligator1.map_fulfills_specific_equation}
  TODO
\end{theorem}

\begin{proof}
  \leanok
  This obviously follows from what we did so far.
\end{proof}

\begin{definition}[Decoding Function ϕ]
  \label{def:Elligator1.ϕ}
  \lean{Elligator1.ϕ}
  \uses{def:Elligator1.x, def:Elligator1.y, thm:Elligator1.map_fulfills_curve_equation}
  \leanok
  In the situation of Theorem 1, the \textbf{decoding function} for the complete Edwards Curve $E : x^2 + y^2 = 1 + d x^2 y^2$ is the function $\phi : \text{F}_q \to E(\text{F}_q)$ defined as follows:

  $\phi(\pm 1) = (0, 1)$; $\text{if } t \notin \{\pm 1\}, \quad \text{then} \quad \phi(t) = (x, y)$.
\end{definition}

\begin{theorem}[Preimages of ϕ]
  \label{thm:Elligator1.ϕ_inv_only_two_specific_preimages}
  \lean{Elligator1.ϕ_inv_only_two_specific_preimages}
  \uses{def:Elligator1.ϕ}
  TODO
\end{theorem}

\begin{definition}[Elliptic Curve over Finite Field]
  \label{def:Elligator1.E_over_F}
  \lean{Elligator1.E_over_F}
  \leanok
  \uses{def:Elligator1.d}
  TODO
\end{definition}

\begin{theorem}[ϕ(F sub q) Property 1]
  \label{thm:Elligator1.ϕ_over_F_prop1}
  \lean{Elligator1.ϕ_over_F_prop1}
  \uses{def:Elligator1.y, def:Elligator1.E_over_F}
  TODO
\end{theorem}

\begin{definition}[Inverted Map Variable η]
  \label{def:Elligator1.η}
  \lean{Elligator1.η}
  \leanok
  \uses{def:Elligator1.d, def:Elligator1.E_over_F}
  TODO
\end{definition}

\begin{theorem}[ϕ(F sub q) Property 2]
  \label{thm:Elligator1.ϕ_over_F_prop2}
  \lean{Elligator1.ϕ_over_F_prop2}
  \uses{def:Elligator1.η, def:Elligator1.r, def:Elligator1.E_over_F}
  TODO
\end{theorem}

\begin{theorem}[ϕ(F sub q) Property 3]
  \label{thm:Elligator1.ϕ_over_F_prop3}
  \lean{Elligator1.ϕ_over_F_prop3}
  \uses{def:Elligator1.η, def:Elligator1.r, def:Elligator1.c, def:LegendreSymbol.χ, def:Elligator1.E_over_F}
  TODO
\end{theorem}

\begin{definition}[Inverted Map Set ϕ(F sub q)]
  \label{def:Elligator1.ϕ_over_F}
  \lean{Elligator1.ϕ_over_F}
  \uses{thm:Elligator1.ϕ_over_F_prop1, thm:Elligator1.ϕ_over_F_prop2, thm:Elligator1.ϕ_over_F_prop3, def:Elligator1.E_over_F}
  TODO
\end{definition}

\begin{theorem}[Point on Curve fulfilling properties also in ϕ(F sub q)]
  \label{thm:Elligator1.point_in_E_over_F_with_props_iff_point_in_ϕ_over_F}
  \lean{Elligator1.point_in_E_over_F_with_props_iff_point_in_ϕ_over_F}
  \uses{thm:Elligator1.ϕ_over_F_prop1, thm:Elligator1.ϕ_over_F_prop2, thm:Elligator1.ϕ_over_F_prop3, def:Elligator1.E_over_F, def:Elligator1.ϕ_over_F}
  TODO
\end{theorem}

\begin{definition}[Inverted Map Variable X2]
  \label{def:Elligator1.X2}
  \lean{Elligator1.X2}
  \uses{def:Elligator1.η, def:Elligator1.E_over_F, def:Elligator1.r}
  TODO
\end{definition}

\begin{definition}[Inverted Map Variable z]
  \label{def:Elligator1.z}
  \lean{Elligator1.z}
  \uses{def:Elligator1.c, def:Elligator1.x , def:Elligator1.E_over_F, def:Elligator1.X2, def:LegendreSymbol.χ}
  TODO
\end{definition}

\begin{definition}[Inverted Map Variable u2]
  \label{def:Elligator1.u2}
  \lean{Elligator1.u2}
  \uses{def:Elligator1.X2, def:Elligator1.z}
  TODO
\end{definition}

\begin{definition}[Inverted Map Variable t2]
  \label{def:Elligator1.t2}
  \lean{Elligator1.t2}
  \uses{def:Elligator1.u2}
  TODO
\end{definition}

\begin{theorem}[Inverted Map Point to Representative]
  \label{thm:Elligator1.invmap_representative}
  \lean{Elligator1.invmap_representative}
  \uses{def:Elligator1.ϕ, def:Elligator1.t2, def:Elligator1.ϕ_over_F, def:Elligator1.E_over_F}
  TODO
\end{theorem}

\begin{definition}[Binary Digits b]
  \label{def:Elligator1.b}
  \lean{Elligator1.b}
  TODO
\end{definition}

\begin{definition}[Set of Potential Representatives S]
  \label{def:Elligator1.S}
  \lean{Elligator1.S}
    \uses{def:Elligator1.b}
  TODO
\end{definition}

\begin{definition}[Binary to Natural Number Function σ]
  \label{def:Elligator1.σ}
  \lean{Elligator1.σ}
    \uses{def:Elligator1.b}
  TODO
\end{definition}

\begin{definition}[Injective Map ι]
  \label{def:Elligator1.ι}
  \lean{Elligator1.ι}
  \uses{def:Elligator1.σ, def:Elligator1.ϕ}
  TODO
\end{definition}

\begin{theorem}[Cardinality of S]
  \label{thm:Elligator1.S_cardinality}
  \lean{Elligator1.S_cardinality}
  \uses{def:Elligator1.S}
  TODO
\end{theorem}

\begin{theorem}[ι is Injective Map]
  \label{thm:Elligator1.ι_injective_map}
  \lean{Elligator1.ι_injective_map}
  \uses{def:Elligator1.ι}
  TODO
\end{theorem}

\begin{definition}[Set ι of S]
  \label{def:Elligator1.ι_S}
  \lean{Elligator1.ι_S}
  \uses{def:Elligator1.ι, def:Elligator1.S}
  TODO
\end{definition}

\begin{theorem}[ι(S) equals ϕ(F sub q)]
  \label{thm:Elligator1.ι_S_eq_ϕ_over_F}
  \lean{Elligator1.ι_S_eq_ϕ_over_F}
  \uses{def:Elligator1.ι, def:Elligator1.ϕ_over_F}
  TODO
\end{theorem}
